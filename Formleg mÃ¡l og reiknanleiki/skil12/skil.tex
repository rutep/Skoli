\documentclass[]{article}
\usepackage[T1]{fontenc}
\usepackage[utf8]{inputenc}
%\usepackage[icelandic]{babel}
\usepackage{caption}
\usepackage{circuitikz}
\usepackage{grffile} 
\usepackage[margin=0.4in]{geometry}
\usepackage{enumitem}
% grffile er pakki sem leifir manni að nota "" til þess að forðast að nota
% nafnið á myndinni með.
\usepackage{graphicx}
% \graphicspath{{images/}} Sýnir undir möppu þar sem myndirnar eru

\usepackage{hyperref}
%fyrirlinka - \url{www.....}
\begin{document}


\title{Formleg mál og reiknanleiki}
\author{Pétur}
\maketitle

\section*{1}
O(n)

\section*{2}

\section*{3}

\subsection*{a)}
The Hamilton path is.

$a \longrightarrow 
 f \longrightarrow
 g \longrightarrow
 e \longrightarrow
 d \longrightarrow
 b \longrightarrow
 c $
 
\subsection*{b)}

Four edges can be removed so that Hamilton path still exist.

\subsection*{c)}
Suppose G has a Hamilton circle. There are two nodes that contradict that G is Hamilton circle. Node b and node e. Either node b or e are always visit it twice. Depending on witch path is taken.

\section*{4}

\begin{enumerate}[label=\textbf{\alph*)}]
	\item Yes it is satisfiable
	\item No it is not satisfiable
	\item Yes it is satisfiable
\end{enumerate}
\end{document}
