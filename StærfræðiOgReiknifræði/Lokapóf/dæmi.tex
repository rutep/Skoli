\documentclass[]{article}
\usepackage[T1]{fontenc}
\usepackage[utf8]{inputenc}
%\usepackage[icelandic]{babel}
\usepackage{caption}
\usepackage{circuitikz}
\usepackage{grffile} 
\usepackage[margin=1in]{geometry}

% grffile er pakki sem leifir manni að nota "" til þess að forðast að nota
% nafnið á myndinni með.
\usepackage{graphicx}
% \graphicspath{{images/}} Sýnir undir möppu þar sem myndirnar eru

\usepackage{hyperref}
%fyrirlinka - \url{www.....}
\begin{document}


\title{Lokapróf}
\maketitle

\section*{Próf 2017 - 1.b)}

a) 
- test \\
* test

vigur x = [x1, x2 ... xár] \\
Fall f(x) er línulegt ef $$ f(\alpha x + \beta y) = \alpha f(x) + \beta f(y) $$
f.alla vigra x,y og tölur $\alpha,\beta$ \\
Fall f er línulegt ef til n-vigur a þ.a. $$ f(x) = a ^{T}x  infeldi $$
i) a = (1,1,1,1,1,0,0,1,1, ... ) fyrsta stak mán annað þrið .... \\

ii) Ólínulegt $$ x=[0,1], y=[1,1], \alpha = 1, \beta = -1 $$
$$ f(x-y) = f([0,-1]) \neg 1f(x) -1f([y]) $$
Niðurstaða 1-2 = -1 ólínulegt

iii) a = (0,0,0,0,0,1,1,0,0 ... ) / 52 \\

iv) a = (1,1, ... ,1, -1,-1 ..., -1, 0 ... 0) jan feb rest \\

- test \\

* test


\begin{center}
	fylki
\end{center}
\[ \left( \begin{array}{ccc}
a & b & c \\
d & e & f \\
g & h & i \end{array} \right)\] 



\end{document}