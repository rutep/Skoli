\documentclass[]{article}
\usepackage[T1]{fontenc}
\usepackage[utf8]{inputenc}
%\usepackage[icelandic]{babel}
\usepackage{caption}
\usepackage{circuitikz}
\usepackage{grffile} 
\usepackage[margin=1in]{geometry}
\usepackage{listings}
\usepackage{color}
\usepackage{mathtools}
% grffile er pakki sem leifir manni að nota "" til þess að forðast að nota
% nafnið á myndinni með.
\usepackage{graphicx}
% \graphicspath{{images/}} Sýnir undir möppu þar sem myndirnar eru

\usepackage{hyperref}
%fyrirlinka - \url{www.....}

\definecolor{mygreen}{rgb}{0,0.6,0}
\definecolor{mygray}{rgb}{0.5,0.5,0.5}
\definecolor{mymauve}{rgb}{0.58,0,0.82}

\lstset{ %
	backgroundcolor=\color{white},   % choose the background color
	basicstyle=\footnotesize,        % size of fonts used for the code
	breaklines=true,                 % automatic line breaking only at whitespace
	captionpos=b,                    % sets the caption-position to bottom
	commentstyle=\color{mygreen},    % comment style
	escapeinside={\%*}{*)},          % if you want to add LaTeX within your code
	keywordstyle=\color{blue},       % keyword style
	stringstyle=\color{mymauve},     % string literal style
}
\begin{document}
	\title{Stærfræði og reiknifræði\\
		Verkefni 5\\
		Pétur Daníel Ámundason\\
	}
	\maketitle
	
	\section*{1.}
	
	\subsection*{a)}
	Svar: Útkoman er skilgreind $ A^{T}A + C = 10 * 10$
	
	\subsection*{b)}
	
	Svar: Útkoman er skilgreind $B*C^{3} = 20 * 10$
	
	\subsection*{c)}
	
	Svar: Útkoman er skilgreind aðeins ef I = $ B^{T}B $ þá er lausninn $I + BC^{T} = 20*10$
	
	\subsection*{d)}
	
	Svar: Útkoman er ekki skilgreind
	
	\subsection*{e)}
	
	Svar: Útkoman er ekki skilgreind
	
	\section*{2.}
	
	\subsection*{10.17}
	
	\subsection*{a)}
	
	Svar: Allir sjúklingar N eru sýktir með öllum sjúkdómum n, N*n.
	% N sjúklingar geta haft n sjúkdóma. N*n fylki S
	% Sij: ef 1 {Sjúklingur i er sýktur af sjúkdómi j}
	% Sij: ef 0 {i er ekki sýktur}
	
	\subsection*{b)}
	
	Svar: i eru orðinn sjúkdómar og j sjúklingar þannig að núna gildir.\\
	
	\[ S_{ij} =
	\begin{cases}
	1       & \quad \text{Sjúkdómur i hefur sýkt sjúkling j}\\
	0	    & \quad \text{Sjúkdómur i hefur ekki sýkt sjúkling j}
	\end{cases}
	\]
	\\
	
	Þannig að S$ ^{T} * 1 $ er þá $ n * N $ fylki og eftir bylltingu þá eru allir sjúdómar búnir að sýkja alla sjúklinga. 
	
	\subsection*{10.31}
	
	\subsection*{a)}
	
	
	
\end{document}