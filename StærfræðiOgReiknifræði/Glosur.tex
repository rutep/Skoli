\documentclass[]{article}
\usepackage[T1]{fontenc}
\usepackage[utf8]{inputenc}
%\usepackage[icelandic]{babel}
\usepackage{caption}
\usepackage{circuitikz}
\usepackage{grffile} 
\usepackage[margin=1in]{geometry}

% grffile er pakki sem leifir manni að nota "" til þess að forðast að nota
% nafnið á myndinni með.
\usepackage{graphicx}
% \graphicspath{{images/}} Sýnir undir möppu þar sem myndirnar eru

\usepackage{hyperref}
%fyrirlinka - \url{www.....}
\begin{document}


\title{Stærfræði og reiknifræði \\
	Glósur}
\author{Pétur Daníel Ámundason \\ \\ pda3@hi.is}
\maketitle

\section*{1.}


Vigur er endanlegur listi af tölum þar sem röð skiptir máli.\\
Lengd vigurs er fjöldi staka í honum \\
n-vigur er vigur með n stökum [1,2,3, ... , n] \\
R táknar mengi rauntalna sem 8 bæta tölur (64 bitar) \\
R$ ^{n} $ ---- allr n-vigrar \\
Ef a er vigur þá táknar a$_{i}$ i-ta stakið vigrar a og b séu eins ef þeir eru jafn langir og öll stök eru eins, þ.e. a$ _{i} $ = b$ _{i} $ f, i = 1, ... , n\\
a $\epsilon$ $\Re^{n}$ táknar að a sé n-vigur með rauntölugildum stökum.
\\
\\
Getum staflað 2 eða fleiri vigrum ef a og b þá er $\begin{array}{|c|} 
	a \\ b
\end{array}
$
\\

Táknum vigra með lágstöfum, t.d. a,b\\
Í núllvigur af lengd n eru öll stök 0.\\
Sundum táknað $ 0_{n} $ en oftast 0\\
þar sem vódd ræðst af samhengi\\

\subsection*{Dæmi}
Ef a er 3-vigur og a=0 \\
\underline{Einnvigur} af lengd n hefur öll stök 1.\\
Stundum táknað 1$ _{n} $ er oftast með 1 \\
\\
\underline{Í einingarvigri (unit vector)} eru öll stök 0 nema eitt sem er 1. I i-ta einingarvigri (e$_{i}  $) er i-ta stak 1.

\subsection*{Dæmi}
e$ _{1} $ = $\begin{array}{|c|} 1 \\ 0 \end{array}$, e$ _{2} $ = $\begin{array}{|c|} 0 \\ 1 \end{array}$
\\
\\
Litir táknaðir með 3-vigri (R,G,B) , (1,0,1) er fjólublár\\\\
\\
\\
Hlutmengi s $\subset$ {T1,T2,T3}\\
Má tákna með 4-vigri þ.s. i-ta stak 1 ef Ti $\epsilon$ S annars 0 .T.d tákna (0,0,1,1) mengið{T3,T4}\\
\\
\\
\pagebreak



\end{document}