\documentclass[]{article}
\usepackage[T1]{fontenc}
\usepackage[utf8]{inputenc}
%\usepackage[icelandic]{babel}
\usepackage{caption}
\usepackage{circuitikz}
\usepackage{grffile} 
\usepackage[margin=1in]{geometry}
\usepackage{listings}
\usepackage{color}
% grffile er pakki sem leifir manni að nota "" til þess að forðast að nota
% nafnið á myndinni með.
\usepackage{graphicx}
% \graphicspath{{images/}} Sýnir undir möppu þar sem myndirnar eru

\usepackage{hyperref}
%fyrirlinka - \url{www.....}

\definecolor{mygreen}{rgb}{0,0.6,0}
\definecolor{mygray}{rgb}{0.5,0.5,0.5}
\definecolor{mymauve}{rgb}{0.58,0,0.82}

\lstset{ %
	backgroundcolor=\color{white},   % choose the background color
	basicstyle=\footnotesize,        % size of fonts used for the code
	breaklines=true,                 % automatic line breaking only at whitespace
	captionpos=b,                    % sets the caption-position to bottom
	commentstyle=\color{mygreen},    % comment style
	escapeinside={\%*}{*)},          % if you want to add LaTeX within your code
	keywordstyle=\color{blue},       % keyword style
	stringstyle=\color{mymauve},     % string literal style
}
\begin{document}
\title{Þýðendur \\
	Þýðandi og milliÞula fyrir NanoMorpho með endurkvæmri ofanferð\\
	Pétur Daníel Ámundason\\
	}
\maketitle

\lstinputlisting[language=java]{NanoMorphoParser.java}
\section*{jflex kóðinn}
\begin{lstlisting}
/**
java -jar JFlex-1.6.0.jar nanolexer.jflex
javac NanoLexer.java
java NanoLexer inntaksskra > uttaksskra
make test
*/

import java.io.*;

%%

%public
%class NanoLexer
%unicode
%byaccj

%{

// Skilgreiningar a tokum (tokens):
final static int ERROR = -1;
final static int IF = 1001;
final static int DEFINE = 1002;
final static int NAME = 1003;
final static int LITERAL = 1004;
final static int WHILE = 1005;
final static int VAR = 1006;
final static int ELSIF = 1007;
final static int RETURN = 1008;
final static int ELSE = 1009;
final static int OPNAME = 1010;


// Breyta sem mun innihalda les (lexeme):
public static String lexeme;

public static void main( String[] args ) throws Exception
{
NanoLexer lexer = new NanoLexer(new FileReader(args[0]));
int token = lexer.yylex();
while( token!=0 )
{
System.out.println(""+token+": \'"+lexeme+"\'");
token = lexer.yylex();
}
}

%}

/* Reglulegar skilgreiningar */

/* Regular definitions */

_DIGIT=[0-9]
_FLOAT={_DIGIT}+\.{_DIGIT}+([eE][+-]?{_DIGIT}+)?
_INT={_DIGIT}+
_STRING=\"([^\"\\]|\\b|\\t|\\n|\\f|\\r|\\\"|\\\'|\\\\|(\\[0-3][0-7][0-7])|\\[0-7][0-7]|\\[0-7])*\"
_CHAR=\'([^\'\\]|\\b|\\t|\\n|\\f|\\r|\\\"|\\\'|\\\\|(\\[0-3][0-7][0-7])|(\\[0-7][0-7])|(\\[0-7]))\'
_DELIM=[={},()\[\];]
_NAME=([:letter:]+{_DIGIT}*)
_OPNAME=[\+\-*/!%&=><\^\|\|]+

%%

/* Lesgreiningarreglur */

{_DELIM} {
lexeme = yytext();
return yycharat(0);
}

{_OPNAME} {
lexeme = yytext();
return OPNAME;
}

{_STRING} | {_FLOAT} | {_CHAR} | {_INT} | null | true | false {
lexeme = yytext();
return LITERAL;
}

"return" {
lexeme = yytext();
return RETURN;
}

"else" {
lexeme = yytext();
return ELSE;
}

"elsif" {
lexeme = yytext();
return ELSIF;
}

"while" {
lexeme = yytext();
return WHILE;
}

"if" {
lexeme = yytext();
return IF;
}

"define" {
lexeme = yytext();
return DEFINE;
}

"var" {
lexeme = yytext();
return VAR;
}

{_NAME} {
lexeme = yytext();
return NAME;
}

";;;".*$ {
}

[ \t\r\n\f] {
}

. {
lexeme = yytext();
return ERROR;
}

\end{lstlisting}

\end{document}