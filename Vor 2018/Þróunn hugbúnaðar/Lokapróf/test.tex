\documentclass[]{article}
\usepackage[T1]{fontenc}
\usepackage[utf8]{inputenc}
%\usepackage[icelandic]{babel}
\usepackage{caption}
\usepackage{circuitikz}
\usepackage{grffile} 
\usepackage[margin=1in]{geometry}
\usepackage[icelandic]{babel}


% grffile er pakki sem leifir manni að nota "" til þess að forðast að nota
% nafnið á myndinni með.
\usepackage{graphicx}
% \graphicspath{{images/}} Sýnir undir möppu þar sem myndirnar eru

\usepackage{hyperref}
%fyrirlinka - \url{www.....}
\begin{document}


\title{Þróunhugbúnaðar \\
Lokaprófs undirbúningur}
\author{Pétur}
\maketitle

\pagebreak

\tableofcontents

\pagebreak

\section{Software Engineer}
\subsection{Definition}
Hugbúnaðarverkfræðingur verður að hafa góð tök á forritun, slunginn í reikniritum og uppbyggingu gagna. Verkfræðingurinn þarf að hafa góð skil á nokkrum hönnunar módelum heldur en að geta forritað þau.
\subsection{Important skills}
\begin{itemize}
	\item Góða samskiftar hæfileika
	\item Getað búið til líkön af flóknum verkefnum
	\item Getað skipulagt og stjórnað vinnu
\end{itemize}
\section{General project work}
\subsection{Reasons for failure}
\subsubsection{Excessive schedule}
Ef það er of mikið vinnuálag á verkefninu sem unnið er að þá getur verkefnið átt í hættu að vera unnið illa.
\subsubsection{Changing needs}
Breytingar í miðju verkefni geta haft slæm áhrif á verkefni sem unnið er að.
\subsubsection{Lack of documented project plan}
Mikilvægt er að hafa allt nánast ritað í stein svo vinnu menn geta verið á sömublaðsíðu.
\subsection{Causes of Software Project Troubles}
\begin{itemize}
	\item Verkefni unnið í nýju software umhverfi
	\item Breyting á viðskiptavinum 
	\item Tími sem fer í að læra á verkefnið
	\item Miskilningur, mismunandi markmið og uppgjöf á verkefni.
\end{itemize}
\section{Software Process Models}
\subsection{Plan-driven}
\subsection{Agile software process models}
\section{Requirements Engineering}
\section{Software Engineer}
\subsection{Definition}
Hugbúnaðarverkfræðingur verður að hafa góð tök á forritun, slunginn í reikniritum og uppbyggingu gagna. Verkfræðingurinn þarf að hafa góð skil á nokkrum hönnunar módelum heldur en að geta forritað þau.
\subsection{Important skills}
\begin{itemize}
	\item Góða samskiftar hæfileika.
	\item Getað búið til líkön af flóknum verkefnum
	\item Getað skipulagt og stjórnað vinnu.
\end{itemize}
\subsubsection{Functional requirement}
\subsubsection{Quality requirement}
\subsubsection{General condition}
\subsubsection{Two conflict example}
\subsection{Technical detail in user story}
\subsubsection{How much technical detail should be in a user story}
\subsubsection{Example: good user story}
\subsubsection{Example: bad user story}
\section{Effort Estimation}
\subsection{Planing poker}
\subsubsection{Based on differing skills and experience, individual team members may estimate different efforts for any requirement. Discuss whether planning poker eliminates this problem}
\subsubsection{Assume your team came up with an effort spread of 8, 13, 40 and 40 for a particular user story. Interpret and deal with the result}
\subsection{Anchoring effect}

\section{Project Planning}
\subsection{Assume your 4-person team completed tasks comprising 20 person-days in a 2-week iteration. Calculate the velocity you should assume when planning the next iteration}
\subsection{Imagine the a client expects more functionality in a release then you will be able to complete until the deadline, based on your effort estimates. Suggest a strategy you could use in this situation}

\section{Object-Oriented Analysis and Design}
\subsection{UML University library diagram}
\subsection{Generalization and specialization seem to be contracting the same concept(inheritance). Explain why both terms are accurate nevertheless}
\subsection{Two key differences between abstract classes and interfaces in java}
\subsection{Explain the difference between aggregation and composition of classes in an object-oriented model}

\section{Object-Oriented Programming}
\subsection{Attributes}
\subsection{Explain why any static methods of java class can access only the static attributes of the class}
\subsection{Class variable}
\subsection{Instance variable}
\subsection{Instance vs Class}
\subsection{Explain how singleton pattern ensures that only one instance of a class can exists in the system}

\section{Testing}
ath prófblað ...

\section{Design Patterns}
\section{Explain the purpose of Proxy pattern, and give an example of a scenario (outside the travel domain) where its use would be beneficial}











\end{document}